\logosection{\faGraduationCap}{教育经历}
\datedline{\textbf{江西师范大学}}{\dateRange{2016.09}{2020.06}}

全日制统招本科 \quad 物联网工程 \hfill 江西 - 南昌

\logosection{\faCogs}{专业技能}

\begin{itemize}[parsep=0.5ex]
      \item 擅长 Next.js 全栈开发,SSR,SEO 优化,前端响应式开发
      \item 擅长前端工程化,熟悉 Webpack/Vite/ESLint 等工具的配置和插件开发
      \item 熟悉 ES6+,TypeScript 类型体操和性能优化
      \item 熟悉 VSCode/Chrome 插件/Adobe CEP 插件开发
      \item 熟悉 Github Actions CI
      \item 擅长使用 AI 辅助开发工具,VSCode/Cursor 资深用户
      \item 熟悉 Nodejs 脚本和后端开发,写过 Dart/Python/Java
      \item 熟悉 Firebase/Redis/Postgresql 数据库
      \item 熟悉 Vercel/Neon/Clerk/Upstash/Clarity/Cloudflare/Amplitude/GCP 等开发平台
    \end{itemize}
\end{itemize}

\logosection{\faSuitcase}{工作经历}

\datedline{\textbf{\href{https://openart.ai}{openart.ai} - 深圳 - 全栈开发工程师}}{\dateRange{2024.04}{2025.02}}
\datedline{\textbf{\href{https://www.gaoding.com/}{稿定设计} - 厦门 - 前端开发工程师}}{\dateRange{2020.06}{2023.10}}
\datedline{\textbf{字节跳动  - 北京 - 前端开发工程师 - 实习}}{\dateRange{2019.07}{2019.10}}


\logosection{\faWrench}{项目经历}
\textbf{\href{https://openart.ai}{openart.ai}}

\vspace{10pt}

核心开发

\begin{itemize}
\item 技术栈:Next.js(Pages Router) + MUI + Firebase
\item 工作内容:
  \begin{itemize}
    \item 基本上所有页面的开发
    \item 首页,生成页用户引导,提高了近 10\% 付费用户转换率
    \item 上新 AI 绘画模型和调整模型选项
    \item 上新 AI app, 例如 Image to 3D
    \item 首页 What's new 弹窗和内部管理页
    \item 图片生成后端 API 重构
    \item 基建优化例如升级 Next12 到 14, 迁移 Yarn1 到 Pnpm, tsc 编译速度优化
    \item SEO 优化,优化了几百万个由于 FCP 和 CLS 指标差影响到 Google 排名的页面
    \item 工作时间外做了很多用户体验优化,例如侧边栏移动端滑动收起,Iconify 图标 SSR
  \end{itemize}
\end{itemize}


\textbf{\href{https://artiffuse.ai}{artiffuse.ai}}

\vspace{10pt}

个人独立开发项目

\begin{itemize}
\item
  技术栈:Nextjs15(App Router) + Trpc + Shadcn/ui + Neon Postgresql + Clerk Auth + Cloudflare R2 + Upstash Redis + Trigger.dev + Vercel Deploy
\item 工作内容:
  \begin{itemize}
    \item 从 0 到 1 独立开发的 AI 图像生成平台
    \item 实现高性能的图片生成系统,高度可扩展模型和参数
    \item 支持暗黑模式,移动端适配,SSR 优化
  \end{itemize}
\end{itemize}

\textbf{融合图片编辑器}

\vspace{10pt}

\begin{itemize}
\item 技术栈:Vue 2.7 + Pinia + TypeScript + Vite + Monorepo + Pnpm + Turborepo
\item 工作内容:
  \begin{itemize}
    \item 迁移图表元素编辑代码从 Vue 2.6 到 2.7
    \item 通过避免重复打包同一个依赖的不同版本和 importmap 等手段优化项目打包体积
    \item 编写项目模版生成器脚本减少新建 app 和 package 成本
    \item 批量处理 lint 错误,避免每次 PR review 需要看一些和业务逻辑无关的修复 lint 的代码
    \item 自己编写了一些实用的 ESLint 规则优化开发体验
    \item 维护 Vite 和 Rollup 打包脚本,实现了一些实用的 Vite 插件,例如检测重复依赖,语义化分块等
    \item 持续优化 CI 体验,支持在 CI 中只 lint 修改的文件,检测是否引进了新的重复依赖
    \item 优化 Turborepo 报错信息帮助开发快速定位构建问题
    \item 编写 npm preinstall 脚本强制统一所有开发本地开发环境,避免因为环境不一致导致无法复现问题
    \item 通过 git post-checkout hook 检测是否切换到不符合规范的分支名,及时给予开发警告
    \item 协助同事解决 VSCode 代码提示,打包,CI 遇到的问题
    \item 编写和持续维护项目文档,帮助同事快速上手项目和解决常见问题
  \end{itemize}
\end{itemize}

\textbf{平面模版 - PS 导出插件}

\vspace{10pt}

\begin{itemize}
\item 技术栈:Adobe CEP + React + Redux Toolkit + TypeScript + Webpack
\item 工作内容:
  \begin{itemize}
    \item 单人完成项目技术选型,项目搭建和所有的业务实现
    \item 优化图层遍历效率提高 50 倍以上
    \item 写了一个简易打包器支持 ExtendScript 打包成单文件
    \item 开源了多个 VSCode 插件提高 PS CEP 插件开发效率
  \end{itemize}
\end{itemize}

\textbf{生产链路工具箱}

\vspace{10pt}

\begin{itemize}
\item 技术栈:Electron + Vue3 + Vuex + Vite
\item 工作内容:
  \begin{itemize}
    \item 负责技术选型和项目搭建
    \item 负责设计实现 Adobe 插件安装卸载逻辑
    \item 负责设计 Adobe 插件 manifest 格式
    \item 负责设计实现工具箱自身的扩展架构,并通过内置扩展实现视频剪辑模版本地调试等能力
    \item 通过自定义协议实现 Adobe 插件唤醒工具箱
    \item 开发体验优化:对 IPC 调用进行封装提供了详细的接口调用和生命周期日志,封装了一个交互式打包脚本
    \item 开源模版:\href{https://github.com/tjx666/electron-vue-vite-boilerplate}{electron-vue-vite-boilerplate}
  \end{itemize}
\end{itemize}

\textbf{视频剪辑模版 - AE 导出插件}

\vspace{10pt}

\begin{itemize}
\item 技术栈:Adobe CEP + React + Redux Toolkit + TypeScript + Webpack
\item 工作内容:
  \begin{itemize}
    \item 带领两个新入职同事共同攻克该业务,负责组内任务分配和技术攻坚
    \item 负责技术选型,项目搭建和开发体验优化
    \item 参考 Lottie 的 AE 导出插件 Bodymovin 设计实现导出流程
    \item 借鉴 ESLint 的 rule 设计实现可扩展的导出校验框架
    \item 负责实现大部分类型的图层到元素,补间动画,文字动画等的转换
    \item 开源 Adobe CEP 插件开发模版:\href{https://github.com/tjx666/cep-react-webpack-boilerplate}{cep-react-webpack-boilerplate}
  \end{itemize}
\end{itemize}

\textbf{移动端-新版拼图拼视频}

\vspace{10pt}

使用公司自研的跨端框架对旧版拼图子应用重写,将多张图片拼成一张图片,支持布局,自由,拼长图模式。

\begin{itemize}
\item 技术栈:前期 Quickjs + TypeScript,后期 Flutter + FFI
\item 工作内容:
  \begin{itemize}
    \item 主要负责交互层和业务需求开发,以及分层架构,手势系统,虚线 Layer 等少部分框架层面的设计和维护
    \item 负责实现编辑器模型导出 JSON
    \item 实现布局模式拉伸限制,拼接模式调节边界,拼接模式停止拉自动居中
    \item 设计实现导出图片尺寸压缩策略,设计实现布局模式切换自由模式变换规则
  \end{itemize}
\end{itemize}

\logosection{\faInstitution}{个人评价}

乐于技术分享,积极参与开源,追求极致的开发体验,敢于抛出问题并愿意主动解决问题。

